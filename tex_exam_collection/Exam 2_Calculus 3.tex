\documentclass[12pt]{article}
\usepackage{amsmath, amssymb}
\usepackage{geometry}
\geometry{margin=1in}

\title{MAC 2313 - Calculus III \\ Second Test}
\date{November 8, 2024}
\author{Instructor: Prof. Kern}

\begin{document}

\maketitle

\noindent
\textbf{Name:} \rule{10cm}{0.4pt}

\vspace{0.5cm}

\noindent
You will have one hour and fifteen minutes to complete this exam. \textbf{YOU MUST SHOW ALL YOUR WORK FOR FULL CREDIT.} The majority of the credit you receive will be based on the completeness and clarity of your responses. Unless stated otherwise, answers should be exact values rather than approximations. Scientific calculators are allowed, along with standard graphing calculators (i.e., TI-84), but not calculators that can do symbolic differentiation/integration (e.g., TI-89, TI-Nspire). No other electronic device is permitted. This test is closed book and closed notes.

\vspace{0.5cm}

\noindent
\begin{tabular}{|c|c|c|}
\hline
Question & Points & Score \\
\hline
1 & 7 & \\
2 & 14 & \\
3 & 7 & \\
4 & 14 & \\
5 & 14 & \\
6 & 16 & \\
7 & 14 & \\
8 & 14 & \\
\hline
Total & 100 & \\
\hline
\end{tabular}

\newpage

\section*{1. (7 points)}
Find \( \frac{\partial^2 f}{\partial x^2} \) given \( f(x, y) = \ln(x^2 + y^2) \)

\section*{2. (14 points)}
Use the Chain Rule to find \( \frac{\partial w}{\partial t} \) given the equations below. Make sure to write your answer in the correct form. (Note: If Chain Rule for partial derivatives is \textbf{NOT} used, at most half credit will be given.)
\[
w(x, y, z) = x^2 y z^3, \quad x(s, t) = s e^{ts}, \quad y(s, t) = st, \quad z(s, t) = st^{-1}
\]

\section*{3. (7 points)}
Sketch the contour map of \( f(x, y) = \sqrt{x^2 + y^2} \), including at least the level curves for 0, 1, 2, and 3.

\section*{4. (14 points)}
Find the limit if it exists, or show it does not exist:
\[
\lim_{(x, y) \to (0, 0)} \frac{x^3}{x^3 + y^3}
\]

\section*{5. (14 points)}
Consider the surface given by the equation \( z = x^2 + y - 5y^2 \). Find the equation of the plane tangent to the surface at the point \( (1, 1, -3) \).

\section*{6. (16 points)}
Find and classify each of the critical points of 
\[
g(x, y) = x^2 + y^2 + 2x^2 y + 3
\]
as a local maximum, local minimum, or saddle point.

\section*{7. (14 points)}
Use Lagrange multipliers to find the minimum value of 
\[
f(x, y, z) = x^2 + y^2 + z^2
\]
given the constraint 
\[
g(x, y, z) = 2x + y + 4z = 6.
\]
(Note: If Lagrange multipliers are \textbf{NOT} used, at most half credit will be given.)

\section*{8. (14 points)}
For this problem, assume \( f(x, y) = x^2 y \).
\begin{enumerate}
    \item[(a)] Find \( \nabla f \).
    \item[(b)] Find the directional derivative of \( f(x, y) \) in the direction of the vector \( \vec{w} = -\vec{i} + \vec{j} \) at the point \( (-1, 4) \).
    \item[(c)] At the point \( (-1, 4) \), find a vector in the direction in which \( f(x, y) \) increases the most rapidly.
    \item[(d)] What is the value of the maximum rate of change of \( f \) at \( (-1, 4) \)?
\end{enumerate}

\end{document}
