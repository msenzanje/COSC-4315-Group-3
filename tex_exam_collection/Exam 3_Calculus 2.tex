MAC 2312 CALCULUS II CRN 83557 MWF
Final Exam
Term Fall, 2024
Full Name:
Instructions
1. Total time: 2 hour 15 minutes.
2. Write the information requested above.
3. Switch off any electronic devices.
4. Calculators are not allowed.
5. Write the solution in the given space.
6. Show all your work for full credit.
7. Scratch papers are provided but will not be graded.
1

Q.N.
Points
Score
1
10
2
10
3
10
4
10
5
10
6
10
7
10
8
10
9
10
Bonus
10
Total
90
2 / 12

1. (10 points) Find the area of the region bounded by the curves:
x = 2y2,
x = 4 + y2
3 / 12

2. (10 points) Find the volume of the solid obtained by rotating the region bounded by
y = x2, y = 2x ;
about the y-axis.
4 / 12

3. (10 points) Determine whether the series is convergent or divergent. If it is convergent, find its sum.
5 + 2
5 + 4
25 +
8
125 + 16
625 + . . .
5 / 12

4. (10 points) Find the average value of the function on the given interval:
g(θ) = cos4 θ sin θ,
[0, π].
6 / 12

5. (10 points) Determine whether the series converges or diverges. Also, write the test you are using.
(a) (5 pts)
∞
X
n=1
n
n3 + 1
(b) (5 pts)
∞
X
n=1
4n
3n −2
7 / 12

6. (10 points) Determine whether the series is absolutely convergent, conditionally convergent, or
divergent.
(a) (4 pts)
∞
X
n=1
(−1)n−1
n4 + 2
(b) (6 pts)
∞
X
n=1
(−1)n
5n + 1
8 / 12

7. (10 points) Find the exact length of the curve:
y = (x + 5)3/2,
0 ≤x ≤3.
9 / 12

8. (10 points) Find the exact area of the surface obtained by rotating the curve about the x-axis:
y = 3 −2x,
0 ≤x ≤2.
10 / 12

9. (10 points) Find the radius of convergence and interval of convergence of the power series.
∞
X
n=0
(x −2)n
n
11 / 12

Bonus Problem (Extra 10 points)
[Use:
1
1 −x =
∞
X
n=0
xn]
(a) (5 points) Find a power series representation for the function and determine the radius of convergence.
f(x) = ln(1 + x)
(b) (5 points) Use part (a) to find a power series for
f(x) = x ln(1 + 2x)
12 / 12

