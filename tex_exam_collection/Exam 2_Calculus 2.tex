\documentclass[12pt]{article}
\usepackage{amsmath, amssymb}
\usepackage{geometry}
\geometry{margin=1in}

\title{MAC 2312 CALCULUS II \\ Exam 2}
\date{Term: Fall, 2024}
\author{CRN 83557 \\ MWF}

\begin{document}

\maketitle

\noindent
\textbf{Full Name:} \rule{10cm}{0.4pt}

\vspace{0.5cm}

\section*{Instructions}
\begin{enumerate}
    \item Total time: 1 hour 15 minutes.
    \item Write the information requested above.
    \item Switch off any electronic devices.
    \item Calculators are not allowed.
    \item Write the solution in the given space.
    \item Show all your work for full credit.
    \item Scratch papers are provided but will not be graded.
\end{enumerate}

\vspace{0.5cm}

\noindent
\begin{tabular}{|c|c|c|}
    \hline
    Q.N. & Points & Score \\
    \hline
    1 & 10 & \\
    2 & 10 & \\
    3 & 10 & \\
    4 & 10 & \\
    5 & 10 & \\
    6 & 10 & \\
    Bonus & 10 & \\
    \hline
    Total & 60 & \\
    \hline
\end{tabular}

\newpage

\section*{1. (10 points)}
Find the exact length of the curve: \\
\[
y = \frac{2}{3}x^{3/2}, \quad 0 \leq x \leq 3
\]

\section*{2. (10 points)}
Find the exact area of the surface obtained by rotating the curve about the x-axis: \\
\[
y = 5 - 2x, \quad 0 \leq x \leq 2
\]

\section*{3. (10 points)}
Find the integral and determine whether the integral is convergent or divergent: \\
\[
\int_0^{\infty} 2x e^{-x^2} \, dx
\]

\section*{4. (10 points)}
Find the centroid of the region bounded by the given curve: \\
\[
y = x + 1, \quad -1 \leq x \leq 1
\]

\section*{5. (10 points)}
Given Parametric curve: \( x = t + 5, \quad y = t^3 - 3t^2 \)
\begin{enumerate}
    \item[(a)] (2 pts) Find the corresponding Cartesian equation by eliminating the parameter.
    \item[(b)] (6 pts) Find derivatives \( \frac{dy}{dx} \) and \( \frac{d^2y}{dx^2} \) in terms of \( t \). \\
    [Hint: \( \frac{dy}{dx} = \frac{dy/dt}{dx/dt} \), and \( \frac{d^2y}{dx^2} = \frac{d(dy/dx)/dt}{dx/dt} \)]
    \item[(c)] (2 pts) For which values of \( t \) is the curve concave upward?
\end{enumerate}

\section*{6. (10 points)}
Find the area of the Polar region that is bounded by the given curve and lies in the specified sector. \\
[Hint: \( \sin 2\theta = 2\sin \theta \cos \theta \)]
\[
r = \sin \theta + \cos \theta, \quad 0 \leq \theta \leq \pi
\]

\section*{Bonus Problem (Extra 10 points)}
\begin{enumerate}
    \item[(a)] (5 points) Determine whether the sequence converges or diverges. If it converges, find the limit: \\
    \[
    a_n = \ln(3n^2 + 1) - \ln(n^2 + n)
    \]
    \item[(b)] (5 points) Find a formula for the general term \( a_n \) of the sequence, assuming that the pattern of the first few terms continues: \\
    \[
    \left\{ \frac{1}{3}, -\frac{4}{5}, \frac{9}{7}, -\frac{16}{9}, \frac{25}{11}, \dots \right\}
    \]
\end{enumerate}

\end{document}
