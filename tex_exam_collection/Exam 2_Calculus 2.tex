MAC 2312 CALCULUS II CRN 83557 MWF
Exam 2
Term Fall, 2024
Full Name:
Instructions
1. Total time: 1 hour 15 minutes.
2. Write the information requested above.
3. Switch off any electronic devices.
4. Calculators are not allowed.
5. Write the solution in the given space.
6. Show all your work for full credit.
7. Scratch papers are provided but will not be graded.
1

Q.N.
Points
Score
1
10
2
10
3
10
4
10
5
10
6
10
Bonus
10
Total
60
2 / 9

1. (10 points) Find the exact length of the curve:
y = 2
3x3/2,
0 ≤x ≤3.
3 / 9

2. (10 points) Find the exact area of the surface obtained by rotating the curve about the x-axis:
y = 5 −2x,
0 ≤x ≤2.
4 / 9

3. (10 points) Find the integral and determine whether the integral is convergent or divergent:
Z ∞
0
2x e−x2 dx.
5 / 9

4. (10 points) Find the centroid of the region bounded by the given curve:
y = x + 1, −1 ≤x ≤1
6 / 9

5. (10 points) Given Parametric curve: x = t + 5,
y = t3 −3t2
(a) (2 pts) Find the corresponding Cartesian equation by eliminating parameter.
(b) (6 pts) Find derivatives dy
dx and d2y
dx2 in terms of ‘t’.
[Hint: dy
dx = dy/dt
dx/dt and d2y
dx2 = d(dy/dx)/dt
dx/dt
]
(c) (2 pts) For which values of t is the curve concave upward ?
7 / 9

6. (10 points) Find the area of the Polar region that is bounded by the given curve and lies in the specified
sector. [Hint: sin 2θ = 2 sin θ cos θ]
r = sin θ + cos θ,
0 ≤θ ≤π
8 / 9

Bonus Problem (Extra 10 points)
(a) (5 points) Determine whether the sequence converges or diverges. If it converges, find the limit:
an = ln (3n2 + 1) −ln (n2 + n).
(b) (5 points) Find a formula for the general term an of the sequence, assuming that the pattern of the
first few terms continues.
1
3, −4
5, 9
7, −16
9 , 25
11, . . .

9 / 9

