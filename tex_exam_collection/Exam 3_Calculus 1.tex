MAC 2311 CALCULUS I CRN 85503
Final Exam
Term Fall, 2024
Full Name:
Instructions
1. Total time: 2 hour 15 minutes.
2. Write the information requested above.
3. Switch off any electronic devices.
4. Calculators are not allowed.
5. Write the solution in the given space.
6. Show all your work for full credit.
7. Scratch papers are provided but will not be graded.
1

Q.N.
Points
Score
1
10
2
10
3
10
4
10
5
10
6
10
7
10
8
10
9
10
10
10
Bonus
10
Total
100
2 / 13

1. (10 points) Find the derivative of the following functions (No need to simplify):
(a) f(x) = \sqrt{x} · tan (x2 + 1)
(b) g(t) =
cos t
3 + 2 sin t
3 / 13

2. (10 points) Find the limit using l’Hospital’s Rule (where you can apply).
(a) \lim_{{x \to 0}}
ex −1 −x
x2
(b)
\lim_{{x \to ∞}}
\ln{x}
x
4 / 13

3. (10 points) Find the derivative dy
dx (or y′) by implicit differentiation.
xey = x −y
5 / 13

4. (10 points) f(x) = x3 −3x2 + 5
(a) (7 pts) Find the intervals on which f is increasing or decreasing.
(b) (3 pts) Find the local maximum and minimum values of f using First Derivative Test.
6 / 13

5. (10 points) Find the dimensions of a rectangle with area 400 m2 whose perimeter is as small as possible.
(Hint: Area = length × width, Perimeter = 2 (length) + 2 (width))
7 / 13

6. (10 points) Find the function f(x) using anti-derivatives.
f′′′(x) = sin x,
f(0) = 2, f′(0) = 1, f′′(0) = 3
8 / 13

7. (10 points) Find the below integrals.
(a) Indefinite Integral:
Z 5x2 + 2 + \sqrt{x}
x
dx
(b) Definite Integral:
Z 3
−1
(ex −π) dx
9 / 13

8. (10 points) Two cars start moving from the same point. One travels north at 3 mi/h and the other travels
east at 4 mi/h. At what rate is the distance between the cars increasing one hour later?
(Hint: Pythagorean theorem: (dist)2 = x2 + y2)
10 / 13

9. (10 points)
(a) (5 pts) Use the Fundamental Theorem of Calculus to find the derivative of the function f′(x).
f(x) =
Z tan x
0
t e−t2 dt
(b) (5 pts) If
Z 7
1
f(x) dx = 10 and
Z 7
1
g(x) dx = 5,
find
Z 7
1
[3f(x) −2g(x)] dx.
11 / 13

10. (10 points) f(x) = 1 + 3x2 −2x3
(a) (5 pts) Find the inflection points of f, and the intervals on which f is concave upward or concave
downward.
(b) (5 pts) Find the local maximum and minimum values of f using the Second Derivative Test.
12 / 13

Bonus Problem (Extra 10 points)
(a) Find the integral by using substitution:
Z
x ex2 dx
(b) Find the integral:
Z 2
−2
(x99 + x3 + x) dx
13 / 13

