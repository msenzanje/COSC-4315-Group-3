\documentclass[12pt]{article}
\usepackage{amsmath, amssymb}
\usepackage{geometry}
\geometry{margin=1in}

\title{MAC 2311 CALCULUS I \\ Exam 2}
\date{Term: Fall, 2024}
\author{}

\begin{document}

\maketitle

\noindent
\textbf{Full Name:} \rule{10cm}{0.4pt}

\vspace{0.5cm}

\section*{Instructions}
\begin{enumerate}
    \item Total time: 1 hour 15 minutes.
    \item Write the information requested above.
    \item Switch off any electronic devices.
    \item Calculators are not allowed.
    \item Write the solution in the given space.
    \item Show all your work for full credit.
    \item Scratch papers are provided but will not be graded.
    \item You are not allowed to use L'Hôpital’s Rule in this exam.
\end{enumerate}

\vspace{0.5cm}

\noindent
\begin{tabular}{|c|c|c|}
    \hline
    Q.N. & Points & Score \\
    \hline
    1 & 10 & \\
    2 & 10 & \\
    3 & 10 & \\
    4 & 10 & \\
    5 & 10 & \\
    6 & 10 & \\
    7 & 10 & \\
    Bonus & 10 & \\
    \hline
    Total & 70 & \\
    \hline
\end{tabular}

\newpage

\section*{1. (10 points)}
Differentiate the below functions using product and quotient rule (No need to simplify):
\begin{enumerate}
    \item[(a)] \( f(x) = (3x^4 - 5e^x)(x^3 + 2x - e^5) \)
    \item[(b)] \( g(x) = \frac{x + \tan x}{1 + \sec x} \)
\end{enumerate}

\section*{2. (10 points)}
Find the derivative of the following functions using Chain rule (No need to simplify):
\begin{enumerate}
    \item[(a)] \( y = e^{\cos x} + \sin(x^2 + 1) \)
    \item[(b)] \( f(t) = \left( \frac{t^2 - 1}{t^2 + 1} \right)^{50} \)
\end{enumerate}

\section*{3. (10 points)}
Find the derivative \( \frac{dy}{dx} \) (or \( y' \)) by implicit differentiation.

\[
x^2 + y^2 = y + 25
\]

\section*{4. (10 points)}
Consider \( y = \frac{x^{5/3} \sqrt{x^2 + 3}}{(x^3 + 1)^8} \)
\begin{enumerate}
    \item[(a)] (3 pts) Use natural log (ln) on both sides and then expand the right side using logarithmic properties.
    \item[(b)] (7 pts) Use logarithmic differentiation to find the derivative of the function \( \frac{dy}{dx} \) (or \( y' \)).
\end{enumerate}

\section*{5. (10 points)}
The radius of a spherical ball is increasing at a rate of 2 ft/sec.
\begin{enumerate}
    \item[(a)] At what rate is the surface area of the ball increasing when the diameter is 8 ft? \\
    (Hint: Surface Area of a sphere \( S = 4\pi r^2 \))
    \item[(b)] At what rate is the volume of the ball increasing when the diameter is 2 ft? \\
    (Hint: Volume of a sphere \( V = \frac{4}{3}\pi r^3 \))
\end{enumerate}

\section*{6. (10 points)}
Let \( f(x) = x^3 - 12x + 1 \)
\begin{enumerate}
    \item[(a)] (4 pts) Find the critical numbers.
    \item[(b)] (6 pts) Find the absolute maximum and absolute minimum values on the closed interval \( [-3, 3] \).
\end{enumerate}

\section*{7. (10 points)}
The edge of a cube is measured to be 2 inches with possible error in measurement 0.01 inches. \\
(Hint: if the edge of a cube is \( x \) then \( S = 6x^2 \) and \( V = x^3 \))
\begin{enumerate}
    \item[(a)] Use differentials to estimate the maximum error in computing the Surface Area \( dS \).
    \item[(b)] Use differentials to estimate the maximum error in computing the Volume \( dV \).
\end{enumerate}

\section*{Bonus Problem (Extra 10 points)}
\begin{enumerate}
    \item[(a)] At what point on the below curve is the tangent horizontal? \\
    \( y = [\ln(x + 3)]^2 \)
    \item[(b)] If \( G(x) = f(3f(2f(x))) \), where \( f(0) = 0 \) and \( f'(0) = 2 \). Find \( G'(0) \).
\end{enumerate}

\end{document}
