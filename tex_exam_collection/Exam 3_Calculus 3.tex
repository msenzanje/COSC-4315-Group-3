MAC 2313 - Calculus III
December 13, 2024
Final Exam
NAME:
You will have one hour and forty-five minutes to complete this exam. YOU MUST SHOW
ALL YOUR WORK FOR FULL CREDIT. The majority of the credit you receive will be
based on the completeness and the clarity of your responses. Unless stated otherwise, answers
should be exact values rather than approximations. Scientific calculators are allowed, along with
standard graphing calculators (i.e., TI-84), but not calculators that can do symbolic differentia-
tion/integration (e.g., TI-89, TI-Nspire). No other electronic device is permitted. This test is closed
book and closed notes.
Question
Points
Score
1
15
2
12
3
10
4
12
5
10
6
12
7
13
8
10
9
15
10
12
11
5
12
12
13
12
14
0
Total:
150
Page 1 of 11
Instructor: Prof. Kern

MAC 2313 - Calculus III
December 13, 2024
1. The curve C is described at every point by the position vector ⃗r(t) = (sin t)ˆı + (t)ˆȷ + (cos t)ˆk.
(a)
(5 points)
⃗T(t), the unit tangent vector
(b)
(5 points)
⃗N(t), the unit normal vector
(c)
(5 points)
Find the tangent line to the space curve at the point when t = π.
Page 2 of 11
Instructor: Prof. Kern

MAC 2313 - Calculus III
December 13, 2024
2. Given points P(1, 3, −1), Q(−1, 5, 0) and R(4, 0, 2) find:
(a)
(3 points)
the vectors ⃗
QP and ⃗
QR.
(b)
(5 points)
a vector orthogonal to the plane determined by P, Q, and R.
(c)
(4 points)
the equation of the plane determined by P, Q, and R.
3.
(10 points)
Let f(x, y) = 6(1 + xy + x2)1/2. Find the linear approximation (i.e., the linearization) of f
around the point (2,2) and use it to find the approximate value of f(2.1, 1.8).
Page 3 of 11
Instructor: Prof. Kern

MAC 2313 - Calculus III
December 13, 2024
4.
(12 points)
Suppose the acceleration of a hang glider is given by
⃗a(t) = −3 cos t ˆı −3 sin t ˆȷ + 2 ˆk
(a) Find the velocity of the hang glider, given that the initial velocity is ⃗v(0) = 2ˆı + 3ˆȷ + ˆk.
(b) Find the position of the hang glider, given that the initial position is ⃗s(0) = 3ˆȷ.
Page 4 of 11
Instructor: Prof. Kern

MAC 2313 - Calculus III
December 13, 2024
5.
(10 points)
Show that the following limit does not exist.
lim
(x,y)→(0,0)
xy
x2 + y2
6.
(12 points)
Determine if the two planes below are parallel, perpendicular or neither. If neither, find the
cosine of the angle between the planes.
2x −y + z = 2
x + y + z = 3
Page 5 of 11
Instructor: Prof. Kern

MAC 2313 - Calculus III
December 13, 2024
7.
(a)
(10 points)
Let w = f(x, y, z) = x2y + x
\sqrt{1} + z and let the point P have the coordinates (1, 2,
3).
Calculate the directional derivative of f(x, y, z) at the point P in the direction of
⃗v = ⟨2, 1, −2⟩.
(b)
(3 points)
At the point P, in what direction is the maximum rate of change?
Page 6 of 11
Instructor: Prof. Kern

MAC 2313 - Calculus III
December 13, 2024
8.
(10 points)
(a) Given w = w(x, y, z), x = x(s, t), y = y(s, t), and z = z(s, t), write out the chain rule for
∂w
∂t .
(b) Let w = xey + y sin z −cos z with x = 2
\sqrt{t,} y = st + \ln{t,} z = s + 4t. Find ∂w
∂t when t = 1
and s = 0.
Page 7 of 11
Instructor: Prof. Kern

MAC 2313 - Calculus III
December 13, 2024
9.
(15 points)
Find and classify all critical points of
f(x, y) = x3 −3x −y3 + 12y
Page 8 of 11
Instructor: Prof. Kern

MAC 2313 - Calculus III
December 13, 2024
10.
(12 points)
Given the integral.
Z 2
0
Z 1
y/2
ex2 dx dy
(a) Sketch the region of integration
(b) Change the order of integration and evaluate.
Page 9 of 11
Instructor: Prof. Kern

MAC 2313 - Calculus III
December 13, 2024
11.
(5 points)
Find the Jacobian of the transformation x = u + 2v, y = 3u + 5v
12.
(12 points)
Using polar coordinates, evaluate
ZZ
D
(x + y) dA
where D is the region in the upper half of the xy-plane bounded by x2+y2 = 4 and x2+y2 = 16.
Page 10 of 11
Instructor: Prof. Kern

MAC 2313 - Calculus III
December 13, 2024
13.
(12 points)
Use a triple integral to find the volume of the tetrahedron in the first octant bounded by the
coordinate planes x = 0, y = 0, and z = 0, and the plane x + 2y + z = 2.
14.
(1 (bonus))
What is a space curve?
Page 11 of 11
Instructor: Prof. Kern

