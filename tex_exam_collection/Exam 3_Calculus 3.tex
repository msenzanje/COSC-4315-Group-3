s\documentclass[12pt]{article}
\usepackage{amsmath, amssymb, graphicx}
\usepackage{fancyhdr}
\usepackage[margin=1in]{geometry}
\pagestyle{fancy}
\fancyhf{}
\rhead{Instructor: Prof. Kern}
\lhead{MAC 2313 - Calculus III}
\rfoot{Page \thepage}

\begin{document}

\begin{center}
    \Large \textbf{MAC 2313 - Calculus III}\\
    \large Final Exam\\
    \large December 13, 2024\\[1em]
    \normalsize \textbf{NAME:} \underline{\hspace{10cm}}\\[2em]
\end{center}

\noindent\textbf{Instructions:} You will have one hour and forty-five minutes to complete this exam. \textbf{YOU MUST SHOW ALL YOUR WORK FOR FULL CREDIT.} The majority of the credit you receive will be based on the completeness and the clarity of your responses. Unless stated otherwise, answers should be exact values rather than approximations. Scientific calculators are allowed, along with standard graphing calculators (i.e., TI-84), but not calculators that can do symbolic differentiation/integration (e.g., TI-89, TI-Nspire). No other electronic device is permitted. This test is closed book and closed notes.

\vspace{2em}

\begin{tabular}{|c|c|c|}
    \hline
    \textbf{Question} & \textbf{Points} & \textbf{Score} \\
    \hline
    1 & 15 & \\
    2 & 12 & \\
    3 & 10 & \\
    4 & 12 & \\
    5 & 10 & \\
    6 & 12 & \\
    7 & 13 & \\
    8 & 10 & \\
    9 & 15 & \\
    10 & 12 & \\
    11 & 5 & \\
    12 & 12 & \\
    13 & 12 & \\
    14 & 0 (Bonus) & \\
    \hline
    \textbf{Total} & \textbf{150} & \\
    \hline
\end{tabular}

\newpage

\section*{1.}
The curve $C$ is described by the position vector $\vec{r}(t) = (\sin t)\hat{\imath} + t\hat{\jmath} + (\cos t)\hat{k}$.

\begin{itemize}
    \item[(a)] (5 points) Find $\vec{T}(t)$, the unit tangent vector.
    \item[(b)] (5 points) Find $\vec{N}(t)$, the unit normal vector.
    \item[(c)] (5 points) Find the tangent line to the space curve at the point when $t = \pi$.
\end{itemize}

\newpage

\section*{2.}
Given points $P(1, 3, -1)$, $Q(-1, 5, 0)$ and $R(4, 0, 2)$, find:

\begin{itemize}
    \item[(a)] (3 points) The vectors $\vec{QP}$ and $\vec{QR}$.
    \item[(b)] (5 points) A vector orthogonal to the plane determined by $P$, $Q$, and $R$.
    \item[(c)] (4 points) The equation of the plane determined by $P$, $Q$, and $R$.
\end{itemize}

\section*{3.}
(10 points) Let $f(x, y) = {6(1 + xy + x^2)}^{1/2$}. Find the linear approximation (i.e., the linearization) of $f$ around the point $(2, 2)$ and use it to find the approximate value of $f(2.1, 1.8)$.

\newpage

\section*{4.}
(12 points) Suppose the acceleration of a hang glider is given by
\[
\vec{a}(t) = -3\cos t \, \hat{\imath} - 3\sin t \, \hat{\jmath} + 2 \hat{k}
\]

\begin{itemize}
    \item[(a)] Find the velocity, given initial velocity $\vec{v}(0) = 2\hat{\imath} + 3\hat{\jmath} + \hat{k}$.
    \item[(b)] Find the position, given initial position $\vec{s}(0) = 3\hat{\jmath}$.
\end{itemize}

\newpage

\section*{5.}
(10 points) Show that the following limit does not exist:
\[
\lim_{(x,y)\to(0,0)} \frac{xy}{x^2 + y^2}
\]

\section*{6.}
(12 points) Determine if the two planes are parallel, perpendicular, or neither. If neither, find the cosine of the angle between the planes:

\[
2x - y + z = 2 \quad \text{and} \quad x + y + z = 3
\]

\newpage

\section*{7.}
\begin{itemize}
    \item[(a)] (10 points) Let $w = f(x, y, z) = x^2y + x\sqrt{1 + z}$ and let the point  $P$ have the coordinates $(1, 2, 3)$. Find the directional derivative of $f(x,y,z)$ at the point $P$ in the direction of $\vec{v} = \langle 2, 1, -2 \rangle$.
    \item[(b)] (3 points) At $P$, in what direction is the maximum rate of change?
\end{itemize}

\newpage

\section*{8.}
\begin{itemize}
    \item[(a)] (5 points) Given $w = w(x, y, z)$, $x = x(s, t)$, $y = y(s, t)$, $z = z(s, t)$, write out the chain rule for $\dfrac{\partial w}{\partial t}$.
    \item[(b)] (5 points) Let $w = x e^y + y\sin z - \cos z$, with $x = 2\sqrt{t}$, $y = st + \ln t$, $z = s + 4t$. Find $\dfrac{\partial w}{\partial t}$ when $t = 1$ and $s = 0$.
\end{itemize}

\newpage

\section*{9.}
(15 points) Find and classify all critical points of:
\[
f(x, y) = x^3 - 3x - y^3 + 12y
\]

\newpage

\section*{10.}
(12 points) Evaluate the double integral:
\[
\int_0^2 \int_{y/2}^1 e^{x^2} \, dx\, dy
\]
\begin{itemize}
    \item[(a)] Sketch the region of integration.
    \item[(b)] Change the order of integration and evaluate.
\end{itemize}

\newpage

\section*{11.}\text
(5 points) Find the Jacobian of the transformation: $x = u + 2v$, $y = 3u + 5v$.

\section*{12.}
(12 points) Using polar coordinates, evaluate:
\[
\iint_D (x + y) \, dA
\]
Where $D$ is the region in the upper half of the $xy$-plane bounded by $x^2 + y^2 = 4$ and $x^2 + y^2 = 16$.

\newpage

\section*{13.}
(12 points) Use a triple integral to find the volume of the tetrahedron in the first octant bounded by the coordinate planes $x = 0$, $y = 0$, $z = 0$, and the plane $x + 2y + z = 2$.

\section*{14.}
(0 points, Bonus) What is a space curve?

\end{document}
